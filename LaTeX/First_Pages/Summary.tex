\section*{German Summary}

In dieser Arbeit wird die Klasse der Bundle-Methoden im Hinblick auf nichtkonvexe Zielfunktionen, inexakte Funktionen- und Subgradientenauswertungen und die M\"oglichkeit der Nutzung von Kr\"ummungsinformation untersucht.
Bundle Methoden werden sehr erfolgreich zur Optimierung von nichtdifferenzierbaren Funktionen eingesetzt, vor allem die Anwendung unter Inexaktheit ist jedoch eine neuere Entwicklung.

Es werden zun\"achst verschiedene Strategien vorgestellt, die es m\"oglich machen, Bundle Algorithmen zur L\"osung nichtkonvexer und nicht exakter Funktionen zu verwenden.
Ein wichtiger Teil der Arbeit besteht au�erdem in der Entwicklung einer Variante der Bundle Methode, die auch eventuell vorhandene Kr\"ummungsinformationen der Zielfunktion nutzt. 
Ein Vergleich der Methode mit einem proximalen Bundle Algorithmus zeigt das Potential dieser Variante auf, welches jedoch je nach Anwendung stark variiert.

Schlie�lich wird noch die Anwendbarkeit von Bundle Methoden auf Bilevel Probleme betrachtet. Dazu wird ein Bundle Algorithmus auf das Problem der Parameteroptimierung f\"ur einen Support Vector Klassifizierer angewendet.