\section{Introduction}

There exists a sound and board theory of classical nonlinear optimization. However, this theory puts strong differentiability requirements on the given problem. Requirements that cannot always be fulfilled in practice.
Examples for such application reach from problems in physics and mechanical engineering \cite{Clarke1990} over optimal control problems up to data analysis \cite{Bagirov2014} and machine learning \cite{Smola2007}.
Other possible fields of applications are risk management and financial calculations \cite{Nesterov2016,Teo2010}. 
Additionally there exist so called stiff problems that are theoretically smooth but numerically nonsmooth due to rapid changes in the gradient \cite{Maekelae1992}. \\
There exists therefore a need for nonsmooth, that is not necessarily differentiable, optimization algorithms. A lot of the underlying theory and was developed in the 1970's, also driven by the ``First World Conference on Nonsmooth Optimization'' taking place in 1977 \cite{Mifflin2012}.
Now, there exists a well understood theoretical framework of nonsmooth analysis to create the basis for practical algorithms. \\
The most popular methods to tackle nonsmooth problems at the moment are bundle methods \cite{Hare2016}. First developed only for convex functions \cite{Hiriart-Urruty1993} the method was soon extended to cope also with nonconvex objective functions \cite{Mifflin1982}.

Additionally the problem of inexact information. 
The most promising algorithms \textcolor{red}{\(\rightarrow\) citation}  in this field are bundle algorithms. \\
\textcolor{red}{from PhD-thesis}
\begin{itemize}
	\item two kinds of algorithms for nonsmooth \\
	subgradient \\
	bundle methods \\
	\item short description of history (of both???)
	\item 
\end{itemize}

Write about developement for convex in beginning.
In this thesis different versions of bundle methods that allow for inexact information for both function values and subgradients are examined more closely. \\
First a proximal bundle method \textcolor{red}{\(Difference between different regularizations explained before\)}...
large-scale optimization: a metric bundle method instead of a proximal bundle method -> limited memory approach \\
Organization as follows: introduction of the most important definitions and results of nonsmooth analysis. Then the introduction of a very basic bundle algorithm which is then generalized for nonconvex functions with nonsmooth optimization. \\
Throughout study of this algorithm including comparison to other approaches to tackle inexact information. \\
Introduction of variable metric (bundle) algorithm to tackle large-scale applications. ``discussion'' how far this is compatible with inexactness. \\
Numerical testing \\ 
discussion \\


